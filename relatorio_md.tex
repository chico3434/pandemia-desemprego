% Options for packages loaded elsewhere
\PassOptionsToPackage{unicode}{hyperref}
\PassOptionsToPackage{hyphens}{url}
%
\documentclass[
]{article}
\usepackage{lmodern}
\usepackage{amssymb,amsmath}
\usepackage{ifxetex,ifluatex}
\ifnum 0\ifxetex 1\fi\ifluatex 1\fi=0 % if pdftex
  \usepackage[T1]{fontenc}
  \usepackage[utf8]{inputenc}
  \usepackage{textcomp} % provide euro and other symbols
\else % if luatex or xetex
  \usepackage{unicode-math}
  \defaultfontfeatures{Scale=MatchLowercase}
  \defaultfontfeatures[\rmfamily]{Ligatures=TeX,Scale=1}
\fi
% Use upquote if available, for straight quotes in verbatim environments
\IfFileExists{upquote.sty}{\usepackage{upquote}}{}
\IfFileExists{microtype.sty}{% use microtype if available
  \usepackage[]{microtype}
  \UseMicrotypeSet[protrusion]{basicmath} % disable protrusion for tt fonts
}{}
\makeatletter
\@ifundefined{KOMAClassName}{% if non-KOMA class
  \IfFileExists{parskip.sty}{%
    \usepackage{parskip}
  }{% else
    \setlength{\parindent}{0pt}
    \setlength{\parskip}{6pt plus 2pt minus 1pt}}
}{% if KOMA class
  \KOMAoptions{parskip=half}}
\makeatother
\usepackage{xcolor}
\IfFileExists{xurl.sty}{\usepackage{xurl}}{} % add URL line breaks if available
\IfFileExists{bookmark.sty}{\usepackage{bookmark}}{\usepackage{hyperref}}
\hypersetup{
  pdftitle={Relatório pandemia e desemprego},
  pdfauthor={CAGIDO, A. C. F; MANGABEIRA, E. F; DO AMARAL, F. R. K; DA MOTA, I. P; CRUZ, J. V. M; MAGALHÃES, J. C. S},
  hidelinks,
  pdfcreator={LaTeX via pandoc}}
\urlstyle{same} % disable monospaced font for URLs
\usepackage[margin=1in]{geometry}
\usepackage{color}
\usepackage{fancyvrb}
\newcommand{\VerbBar}{|}
\newcommand{\VERB}{\Verb[commandchars=\\\{\}]}
\DefineVerbatimEnvironment{Highlighting}{Verbatim}{commandchars=\\\{\}}
% Add ',fontsize=\small' for more characters per line
\usepackage{framed}
\definecolor{shadecolor}{RGB}{248,248,248}
\newenvironment{Shaded}{\begin{snugshade}}{\end{snugshade}}
\newcommand{\AlertTok}[1]{\textcolor[rgb]{0.94,0.16,0.16}{#1}}
\newcommand{\AnnotationTok}[1]{\textcolor[rgb]{0.56,0.35,0.01}{\textbf{\textit{#1}}}}
\newcommand{\AttributeTok}[1]{\textcolor[rgb]{0.77,0.63,0.00}{#1}}
\newcommand{\BaseNTok}[1]{\textcolor[rgb]{0.00,0.00,0.81}{#1}}
\newcommand{\BuiltInTok}[1]{#1}
\newcommand{\CharTok}[1]{\textcolor[rgb]{0.31,0.60,0.02}{#1}}
\newcommand{\CommentTok}[1]{\textcolor[rgb]{0.56,0.35,0.01}{\textit{#1}}}
\newcommand{\CommentVarTok}[1]{\textcolor[rgb]{0.56,0.35,0.01}{\textbf{\textit{#1}}}}
\newcommand{\ConstantTok}[1]{\textcolor[rgb]{0.00,0.00,0.00}{#1}}
\newcommand{\ControlFlowTok}[1]{\textcolor[rgb]{0.13,0.29,0.53}{\textbf{#1}}}
\newcommand{\DataTypeTok}[1]{\textcolor[rgb]{0.13,0.29,0.53}{#1}}
\newcommand{\DecValTok}[1]{\textcolor[rgb]{0.00,0.00,0.81}{#1}}
\newcommand{\DocumentationTok}[1]{\textcolor[rgb]{0.56,0.35,0.01}{\textbf{\textit{#1}}}}
\newcommand{\ErrorTok}[1]{\textcolor[rgb]{0.64,0.00,0.00}{\textbf{#1}}}
\newcommand{\ExtensionTok}[1]{#1}
\newcommand{\FloatTok}[1]{\textcolor[rgb]{0.00,0.00,0.81}{#1}}
\newcommand{\FunctionTok}[1]{\textcolor[rgb]{0.00,0.00,0.00}{#1}}
\newcommand{\ImportTok}[1]{#1}
\newcommand{\InformationTok}[1]{\textcolor[rgb]{0.56,0.35,0.01}{\textbf{\textit{#1}}}}
\newcommand{\KeywordTok}[1]{\textcolor[rgb]{0.13,0.29,0.53}{\textbf{#1}}}
\newcommand{\NormalTok}[1]{#1}
\newcommand{\OperatorTok}[1]{\textcolor[rgb]{0.81,0.36,0.00}{\textbf{#1}}}
\newcommand{\OtherTok}[1]{\textcolor[rgb]{0.56,0.35,0.01}{#1}}
\newcommand{\PreprocessorTok}[1]{\textcolor[rgb]{0.56,0.35,0.01}{\textit{#1}}}
\newcommand{\RegionMarkerTok}[1]{#1}
\newcommand{\SpecialCharTok}[1]{\textcolor[rgb]{0.00,0.00,0.00}{#1}}
\newcommand{\SpecialStringTok}[1]{\textcolor[rgb]{0.31,0.60,0.02}{#1}}
\newcommand{\StringTok}[1]{\textcolor[rgb]{0.31,0.60,0.02}{#1}}
\newcommand{\VariableTok}[1]{\textcolor[rgb]{0.00,0.00,0.00}{#1}}
\newcommand{\VerbatimStringTok}[1]{\textcolor[rgb]{0.31,0.60,0.02}{#1}}
\newcommand{\WarningTok}[1]{\textcolor[rgb]{0.56,0.35,0.01}{\textbf{\textit{#1}}}}
\usepackage{graphicx,grffile}
\makeatletter
\def\maxwidth{\ifdim\Gin@nat@width>\linewidth\linewidth\else\Gin@nat@width\fi}
\def\maxheight{\ifdim\Gin@nat@height>\textheight\textheight\else\Gin@nat@height\fi}
\makeatother
% Scale images if necessary, so that they will not overflow the page
% margins by default, and it is still possible to overwrite the defaults
% using explicit options in \includegraphics[width, height, ...]{}
\setkeys{Gin}{width=\maxwidth,height=\maxheight,keepaspectratio}
% Set default figure placement to htbp
\makeatletter
\def\fps@figure{htbp}
\makeatother
\setlength{\emergencystretch}{3em} % prevent overfull lines
\providecommand{\tightlist}{%
  \setlength{\itemsep}{0pt}\setlength{\parskip}{0pt}}
\setcounter{secnumdepth}{-\maxdimen} % remove section numbering

\title{Relatório pandemia e desemprego}
\author{CAGIDO, A. C. F; MANGABEIRA, E. F; DO AMARAL, F. R. K; DA MOTA, I. P;
CRUZ, J. V. M; MAGALHÃES, J. C. S}
\date{04/08/2020}

\begin{document}
\maketitle

\hypertarget{resumo}{%
\subsection{Resumo}\label{resumo}}

Essa pesquisa tem como propósito tentar achar relações entre a pandemia
o emprego e o desemprego. Fazendo correlações entre as variáveis tentar
entender as mudanças nos trabalhos de quem continuou com emprego,
entender os motivos da demissão por quem perdeu o emprego, verificar
possíveis relaçãoes entre a perda do emprego com a renda, raça, sexo e
idade.

\hypertarget{resultados}{%
\subsection{Resultados}\label{resultados}}

Inicialmente será carregadas as bibliotecas usadas

\begin{Shaded}
\begin{Highlighting}[]
\KeywordTok{library}\NormalTok{(psych)}
\end{Highlighting}
\end{Shaded}

Após carregar as bibliotecas, será lido os dados que estão em um CSV
gerado pelo Google forms

\begin{Shaded}
\begin{Highlighting}[]
\NormalTok{dados <-}\StringTok{ }\KeywordTok{read.csv}\NormalTok{(}\StringTok{'parcial.csv'}\NormalTok{, }\DataTypeTok{encoding =} \StringTok{'UTF-8'}\NormalTok{)}
\end{Highlighting}
\end{Shaded}

O estudo conta com uma amostra de \texttt{97} pessoas.

\hypertarget{idade}{%
\subsubsection{Idade}\label{idade}}

Idade é uma variável quantitativa discreta, e por ter muitas idades
diferentes, será dividida em intervalos.

\begin{Shaded}
\begin{Highlighting}[]
\NormalTok{valores_idade <-}\StringTok{ }\KeywordTok{describe}\NormalTok{(dados}\OperatorTok{$}\NormalTok{Qual.é.a.sua.idade.)}

\NormalTok{sturges <-}\StringTok{ }\KeywordTok{nclass.Sturges}\NormalTok{(dados}\OperatorTok{$}\NormalTok{Qual.é.a.sua.idade.)}
\end{Highlighting}
\end{Shaded}

A tabela de frequência absoluta:

\begin{Shaded}
\begin{Highlighting}[]
\KeywordTok{table}\NormalTok{(}\KeywordTok{cut}\NormalTok{(dados}\OperatorTok{$}\NormalTok{Qual.é.a.sua.idade., }\KeywordTok{seq}\NormalTok{(valores_idade}\OperatorTok{$}\NormalTok{min, valores_idade}\OperatorTok{$}\NormalTok{max, }\DataTypeTok{l =}\NormalTok{ sturges}\OperatorTok{+}\DecValTok{1}\NormalTok{)))}
\end{Highlighting}
\end{Shaded}

\begin{verbatim}
## 
##   (18,22.9] (22.9,27.8] (27.8,32.6] (32.6,37.5] (37.5,42.4] (42.4,47.2] 
##          21          48          12           4           2           2 
## (47.2,52.1]   (52.1,57] 
##           0           2
\end{verbatim}

A tabela de frequência relativa:

\begin{Shaded}
\begin{Highlighting}[]
\KeywordTok{prop.table}\NormalTok{(}\KeywordTok{table}\NormalTok{(}\KeywordTok{cut}\NormalTok{(dados}\OperatorTok{$}\NormalTok{Qual.é.a.sua.idade., }\KeywordTok{seq}\NormalTok{(valores_idade}\OperatorTok{$}\NormalTok{min, valores_idade}\OperatorTok{$}\NormalTok{max, }\DataTypeTok{l =}\NormalTok{ sturges}\OperatorTok{+}\DecValTok{1}\NormalTok{))))}
\end{Highlighting}
\end{Shaded}

\begin{verbatim}
## 
##   (18,22.9] (22.9,27.8] (27.8,32.6] (32.6,37.5] (37.5,42.4] (42.4,47.2] 
##  0.23076923  0.52747253  0.13186813  0.04395604  0.02197802  0.02197802 
## (47.2,52.1]   (52.1,57] 
##  0.00000000  0.02197802
\end{verbatim}

Os valores da variável, como valores de medida central e de dispersão:

\begin{Shaded}
\begin{Highlighting}[]
\NormalTok{valores_idade}
\end{Highlighting}
\end{Shaded}

\begin{verbatim}
##    vars  n  mean   sd median trimmed  mad min max range skew kurtosis   se
## X1    1 97 25.81 6.75     25   24.76 4.45  18  57    39 2.39     7.53 0.69
\end{verbatim}

O histograma com a distribuição por faixa etária:

\begin{Shaded}
\begin{Highlighting}[]
\KeywordTok{hist}\NormalTok{(dados}\OperatorTok{$}\NormalTok{Qual.é.a.sua.idade.)}
\end{Highlighting}
\end{Shaded}

\includegraphics{relatorio_md_files/figure-latex/unnamed-chunk-7-1.pdf}

\hypertarget{sexo}{%
\subsubsection{Sexo}\label{sexo}}

Sexo é uma variável categórica nominal.\\
A tabela de frequência absuluta:

\begin{Shaded}
\begin{Highlighting}[]
\KeywordTok{table}\NormalTok{(dados}\OperatorTok{$}\NormalTok{Qual.é.o.seu.sexo.)}
\end{Highlighting}
\end{Shaded}

\begin{verbatim}
## 
##  Feminino Masculino 
##        56        41
\end{verbatim}

A tabela de frequência relativa:

\begin{Shaded}
\begin{Highlighting}[]
\KeywordTok{prop.table}\NormalTok{(}\KeywordTok{table}\NormalTok{(dados}\OperatorTok{$}\NormalTok{Qual.é.o.seu.sexo.))}
\end{Highlighting}
\end{Shaded}

\begin{verbatim}
## 
##  Feminino Masculino 
## 0.5773196 0.4226804
\end{verbatim}

\hypertarget{conclusuxe3o}{%
\subsection{Conclusão}\label{conclusuxe3o}}

Aqui terá a conclusão.

\end{document}
